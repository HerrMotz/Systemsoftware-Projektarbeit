
\documentclass[runningheads]{llncs}
%
\usepackage[T1]{fontenc}
% T1 fonts will be used to generate the final print and online PDFs,
% so please use T1 fonts in your manuscript whenever possible.
% Other font encodings may result in incorrect characters.
%
\usepackage{graphicx}
% Used for displaying a sample figure. If possible, figure files should
% be included in EPS format.
%
% If you use the hyperref package, please uncomment the following two lines
% to display URLs in blue roman font according to Springer's eBook style:
%\usepackage{color}
%\renewcommand\UrlFont{\color{blue}\rmfamily}
%

% Daniel Motz's packages
\usepackage{amsfonts}
\usepackage{mathtools}
\DeclarePairedDelimiter\ceil{\lceil}{\rceil}
\DeclarePairedDelimiter\floor{\lfloor}{\rfloor}

\usepackage{minted}
% END Daniel Motz's packages

\begin{document}
%
\title{Scheduling on Multicore Systems}
%
%\titlerunning{Abbreviated paper title}
% If the paper title is too long for the running head, you can set
% an abbreviated paper title here
%
\author{Friedrich Answin Daniel Motz}
%
\authorrunning{D. Motz}
% First names are abbreviated in the running head.
% If there are more than two authors, 'et al.' is used.
%
\institute{FMI, Friedrich-Schiller-University Jena, Jena, Germany\\
Seminar Work for Systemsoftware\\
\email{daniel.motz@uni-jena.de}
}
%
\maketitle              % typeset the header of the contribution
%
\begin{abstract}
Please note that the first paragraph of a section or subsection is
not indented. The first paragraph that follows a table, figure,
equation etc. does not need an indent, either.

\keywords{First keyword  \and Second keyword \and Another keyword.}
\end{abstract}
%
%
%
\section{Introduction}
Processor Scheduling is, by definition, the system which holds it all together. It determines which
process is permitted to run, when it may run, and for how long - it is ruler \textit{dei gratia}
and the Computer Architects are it's gods. Some of these gods have left us great works of art,
like the CFS Scheduler (current Linux scheduler) or $O(1)$ Scheduler (previous Linux scheduler), 
but some behaviours of mentioned schedulers seem erratic. New research do scheduling prediction 
based on modelling of CPU schedulers' behaviours~\cite{meehean}.

Moore's law will not be here to protect us from what is inevitable: transistor count will stop
doubling. At some point, transistor sizes will meet atomic boundaries, at which point growth can
solely be realised by parallelisation and clustering. It's questionable whether Quantum Computers
will replace classic computers in scientific calculations, but not completely unimaginable. For
now, one may safely assume that computation will need to rely on classic CPU schedulers for some
years to come. So, there is an undisputed necessity to efficiently manage multicore CPUs, not only
in High Performance Computing, but also desktop PCs.~\cite{thread-sched-mc-platforms}

The key challenges for multicore scheduling, are to achieve the goals formerly set for single-core
CPU schedulers, meaning, that an octa-core processor should behave to a programme, as though it 
was a single-core processor with an octupled performance. With faster CPUs and RAM not growing
equally fast, cycle times to fetch from memory become higher and lie in the hundreds.~\cite{fedorova-phd}
Making use of on-chip caches (all levels of them) is therefore key. Otherwise, overhead introduced
by cache misses will limit the possible effectiveness of multicore CPUs. Cache-affinity, or 
more generally, hardware affinity is a sub-goal of scalability.

Research by Intel from 2007 suggests, that using a runtime called McRT, might enable near linear
scalability for certain tasks, such as recognition, mining and synthesis applications or Xvid.~
\cite{scalability-of-programs-multi-cores}

\section{Parallel Architecture}
\subsection{Motivation}
In Computer Science, researchers 

\section{Notes / TODO}
\subsubsection{Timesharing Schedulers}
\begin{itemize}
    \item Primary goal is low latency for interactive tasks
    \item tasks are divided based on their level of interactivity
    \begin{itemize}
        \item this may be based on multiple factors, such as:
        \item user priority
        \item cpu usage
    \end{itemize}
    \item the tasks are organised into multiple queues depending on their level of interactivity
    \item higher queue / level means higher priority
    \item they move down a queue in two ways: the task does not yield the CPU in a time slice
    \item v
\end{itemize}

\subsubsection{Acknowledgements} Please place your acknowledgments at
the end of the paper, preceded by an unnumbered run-in heading (i.e.
3rd-level heading).

%
% ---- Bibliography ----
%
\bibliographystyle{splncs04}
 
\newpage

\begin{thebibliography}{8}

\bibitem{ostep}
Arpaci-Dusseau, R. H., Arpaci-Dusseau, A. C.: Operating Systems Three Easy Pieces. Version 0.80. Arpaci-Dusseau Books, Inc.,
University of Wisconsin-Madison (2014)

\bibitem{meehean}
Meehean, J. T., : Towards Transparent CPU Scheduling. Dissertation,
University of Wisconsin-Madison (2011)

\bibitem{binary_tree_index_properties}
C. Riesbeck. EECS 311: Trees, \url{https://courses.cs.northwestern.edu/311/html/tree-notes.html}. Last accessed 10. March 2022

\bibitem{thread-sched-mc-platforms}
Rajagopanal, M., Lewis, B. T., Anderson, T. A.: Thread Scheduling for Multi-Core Platforms. Programming Systems Lab, Intel Corporation (2007)

\bibitem{scalability-of-programs-multi-cores}
SAHA, B., ADL-TABATABAI, A., GHULOUM, A., RAJAGOPALAN,
M., HUDSON, R., PETERSEN, L., MENON, V.,
MURPHY, B., SHPEISMAN, T., SPRANGLE, E., ROHILLAH, A.,
CARMEAN, D., AND FANG, J. Enabling scalability and performance
in a large scale CMP environment. EuroSys (March 2007)

\bibitem{fedorova-phd}
FEDOROVA, A. Operating System Scheduling for Chip Multithreaded
Processors. PhD thesis, Division of Engineering and
Applied Sciences, Harvard University, Cambridge, MA, Nov
2006.

\bibitem{ref_article1}
Author, F.: Article title. Journal \textbf{2}(5), 99--110 (2016)

\bibitem{ref_lncs1}
Author, F., Author, S.: Title of a proceedings paper. In: Editor,
F., Editor, S. (eds.) CONFERENCE 2016, LNCS, vol. 9999, pp. 1--13.
Springer, Heidelberg (2016). \doi{10.10007/1234567890}

\bibitem{ref_proc1}
Author, A.-B.: Contribution title. In: 9th International Proceedings
on Proceedings, pp. 1--2. Publisher, Location (2010)
\end{thebibliography}
\end{document}
